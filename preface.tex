Στο τέλος της αναθεώρησης του πρώτου μου βιβλίου, {\it Κώδικας: Και άλλοι Νόμοι του Κυβερνοχώρου (Code: And Other Laws of Cyberspace)}, ο David Pogue, ένας εξαίρετος συγγραφέας και δημιουργός αμέτρητων τεχνικών και σχετικών με τους υπολογιστές κειμένων, έγραψε αυτό:

\begin{quotation}
{\it Σε αντίθεση με την πραγματική νομοθεσία, στο λογισμικό του Διαδικτύου δεν υπάρχει καμία δυνατότητα για τιμωρία. Δεν επηρεάζει τους ανθρώπους που δεν είναι συνδεδεμένοι (και μόνο μια μικρή μειονότητα του παγκοσμίου πληθυσμού είναι). Και αν δεν δε σου αρέσει το πώς λειτουργεί το Διαδίκτυο, μπορείς πάντα να κλείσεις το μόντεμ.} 
\end{quotation}

Ο Progue ήταν σκεπτικός ως προς τον βασικό επιχείρημα του βιβλίου -- το ότι το λογισμικό, ή ο "κώδικας", λειτουργούσαν ως μια μορφή νομοθεσίας -- και η κριτική του εισηγούνταν την χαρούμενη σκέψη πως αν η ζωή στον κυβερνοχώρο ακολουθούσε μια μη επιθυμητή πορεία, θα μπορούσαμε πάντα να κλείσουμε έναν διακόπτη και να γυρίσουμε στο σπίτι. Κλείσε το μόντεμ, αποσύνδεσε τον υπολογιστή, και όποια προβλήματα υπάρχουν σε {\it εκείνον} τον χώρο δε θα μας "επηρέαζαν" πια.

 Ο Progue ίσως να σκεφτόταν σωστά το 1999 -- είμαι δύσπιστος, αλλά μπορεί. Αλλά ακόμα και αν ήταν σωστός τότε, κάτι τέτοιο δεν ισχύει τώρα: ο {\it Ελεύθερος Πολιτισμός} αφορά τα προβλήματα που δημιουργεί το Διαδίκτυο ακόμα και αφού κλείσουμε το μόντεμ. Είναι ένα θέμα που αφορά τις διαμάχες που όλο και εντείνονται σχετικά με το πώς ο κόσμος του Διαδικτύου έχει θεμελιωδώς επηρεάσει τους "ανθρώπους που δεν είναι online". Δεν υπάρχει κανένας διακόπτης που θα μας απομονώσει από τα επιδράσεις του Διαδικτύου.
